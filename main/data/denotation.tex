% !TeX root = ../thuthesis-example.tex

\begin{denotation}[3cm]
  \item[不稳定神经网络] 在文中指代具有梯度爆炸和梯度消失问题的神经网络。
  \item[梯度爆炸和梯度消失] 指在反向传播算法中,梯度可能呈指数级增长或衰减的问题。
  \item[反向传播算法] 指用于训练神经网络的一种常见算法,通过计算梯度来更新网络参数。
  \item[网络层数和复杂度] 表示神经网络的层数和复杂程度,通常与网络的深度和参数数量相关。
  \item[数值不稳定] 指在训练过程中,由于梯度爆炸或梯度消失问题导致的数值不稳定现象。
  \item[模型性能] 指神经网络在任务上的表现,如准确率、收敛速度等。
  \item[李雅普诺夫谱] 用于描述系统动态特性和稳定性的概念。
  \item[李雅普诺夫向量] 用于描述系统特征向量的概念,与李雅普诺夫谱相关。
  \item[伴随李雅普诺夫谱] 指利用李雅普诺夫谱提供的信息来调整梯度传播路径和强度的方法。
  \item[对偶性] 指伴随李雅普诺夫谱方法中的概念,用于解决梯度爆炸问题。
  \item[梯度裁剪] 一种传统的反向传播算法中用于解决梯度爆炸问题的方法,通过限制梯度的范围来控制其大小。
  \item[正则化技术] 另一种传统的反向传播算法中用于解决梯度爆炸问题的方法,通过在损失函数中引入正则化项来限制参数的增长。
  \item[伴随噪声] 本文提出的一种基于伴随李雅普诺夫谱的新反向传播方法,用于调整梯度的传播路径和强度。
  \item[核微分方法] 引入核函数平滑梯度计算的方法,提高计算的稳定性和准确性。
  \item[参数更新] 指在训练过程中,通过梯度下降算法更新神经网络参数的步骤。
  \item[混合优化算法] 结合了伴随噪声和核微分方法的新的优化算法,在处理梯度爆炸问题时具有优势。
  \item[训练速度] 表示神经网络在训练过程中的速度,通常指每个训练样本的处理时间。
  \item[收敛性] 指神经网络在训练过程中是否能够达到最优解的性质。
  \item[最终模型性能] 指训练完成后神经网络在测试数据上的表现。
  \item[神经网络] 指一种由多个神经元组成的计算模型,用于学习和处理复杂的数据关系。
  \item[RNN] 循环神经网络(Recurrent Neural Network)。
  \item[LSTM] 长短期记忆网络(Long Short-Term Memory)。
  \item[GAN] 生成对抗网络(Generative Adversarial Network)。
  \item[RL] 强化学习(Reinforcement Learning)。
  \item[LEs] 李雅普诺夫指数(Lyapunov Exponents)。
  \item[NILSAS] 最小二乘伴随噪声(Noise-Injection Least Squares Adjoint Sensitivity)方法。
  \item[QR 分解] 一种矩阵分解方法,可用于计算李雅普诺夫谱。
\end{denotation}



% 也可以使用 nomencl 宏包,需要在导言区
% \usepackage{nomencl}
% \makenomenclature

% 在这里输出符号说明
% \printnomenclature[3cm]

% 在正文中的任意为都可以标题
% \nomenclature{PI}{聚酰亚胺}
% \nomenclature{MPI}{聚酰亚胺模型化合物,N-苯基邻苯酰亚胺}
% \nomenclature{PBI}{聚苯并咪唑}
% \nomenclature{MPBI}{聚苯并咪唑模型化合物,N-苯基苯并咪唑}
% \nomenclature{PY}{聚吡咙}
% \nomenclature{PMDA-BDA}{均苯四酸二酐与联苯四胺合成的聚吡咙薄膜}
% \nomenclature{MPY}{聚吡咙模型化合物}
% \nomenclature{As-PPT}{聚苯基不对称三嗪}
% \nomenclature{MAsPPT}{聚苯基不对称三嗪单模型化合物,3,5,6-三苯基-1,2,4-三嗪}
% \nomenclature{DMAsPPT}{聚苯基不对称三嗪双模型化合物(水解实验模型化合物)}
% \nomenclature{S-PPT}{聚苯基对称三嗪}
% \nomenclature{MSPPT}{聚苯基对称三嗪模型化合物,2,4,6-三苯基-1,3,5-三嗪}
% \nomenclature{PPQ}{聚苯基喹噁啉}
% \nomenclature{MPPQ}{聚苯基喹噁啉模型化合物,3,4-二苯基苯并二嗪}
% \nomenclature{HMPI}{聚酰亚胺模型化合物的质子化产物}
% \nomenclature{HMPY}{聚吡咙模型化合物的质子化产物}
% \nomenclature{HMPBI}{聚苯并咪唑模型化合物的质子化产物}
% \nomenclature{HMAsPPT}{聚苯基不对称三嗪模型化合物的质子化产物}
% \nomenclature{HMSPPT}{聚苯基对称三嗪模型化合物的质子化产物}
% \nomenclature{HMPPQ}{聚苯基喹噁啉模型化合物的质子化产物}
% \nomenclature{PDT}{热分解温度}
% \nomenclature{HPLC}{高效液相色谱(High Performance Liquid Chromatography)}
% \nomenclature{HPCE}{高效毛细管电泳色谱(High Performance Capillary lectrophoresis)}
% \nomenclature{LC-MS}{液相色谱-质谱联用(Liquid chromatography-Mass Spectrum)}
% \nomenclature{TIC}{总离子浓度(Total Ion Content)}
% \nomenclature{\textit{ab initio}}{基于第一原理的量子化学计算方法,常称从头算法}
% \nomenclature{DFT}{密度泛函理论(Density Functional Theory)}
% \nomenclature{$E_a$}{化学反应的活化能(Activation Energy)}
% \nomenclature{ZPE}{零点振动能(Zero Vibration Energy)}
% \nomenclature{PES}{势能面(Potential Energy Surface)}
% \nomenclature{TS}{过渡态(Transition State)}
% \nomenclature{TST}{过渡态理论(Transition State Theory)}
% \nomenclature{$\increment G^\neq$}{活化自由能(Activation Free Energy)}
% \nomenclature{$\kappa$}{传输系数(Transmission Coefficient)}
% \nomenclature{IRC}{内禀反应坐标(Intrinsic Reaction Coordinates)}
% \nomenclature{$\nu_i$}{虚频(Imaginary Frequency)}
% \nomenclature{ONIOM}{分层算法(Our own N-layered Integrated molecular Orbital and molecular Mechanics)}
% \nomenclature{SCF}{自洽场(Self-Consistent Field)}
% \nomenclature{SCRF}{自洽反应场(Self-Consistent Reaction Field)}
