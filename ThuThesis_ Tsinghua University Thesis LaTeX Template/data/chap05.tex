% !TeX root = ../thuthesis-example.tex

\chapter{总结}

本文回顾了李雅普诺夫谱和李雅普诺夫向量,并介绍了计算李雅普诺夫指数的基本方法和应用。李雅普诺夫指数是用来描述一个动力系统中轨道对初始条件的敏感性的量度。在神经网络中,李雅普诺夫指数可以帮助我们理解网络的稳定性和动态行为。为了计算这些指数,本文采用了QR分解法,这是目前在计算李雅普诺夫谱中最为常用和有效的方法之一。

我们重点分析了在神经网络的训练过程中计算李雅普诺夫谱的表现。实验结果表明,李雅普诺夫指数可以作为一种有效的指标,用于评估网络的稳定性和预测训练过程中可能出现的数值问题。通过对李雅普诺夫指数的分析,我们可以提前发现并解决网络训练中的潜在问题,避免模型在训练后期出现不稳定或发散的现象。

此外,我们还总结了基于伴随阴影的反向传播算法和核微分方法的应用。在理论层面分析了传统反向传播算法在不稳定神经网络中的表现和局限性,强调了梯度爆炸问题对参数更新和模型训练的影响。基于伴随阴影的反向传播方法重新定义了梯度更新规则,并通过实验验证了其在不同类型不稳定神经网络中的有效性。这些方法能够显著减小梯度爆炸的影响,提升模型的收敛速度和性能稳定性。