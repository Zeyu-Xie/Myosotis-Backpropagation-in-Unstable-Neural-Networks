% !TeX root = ../thuthesis-example.tex

% 中英文摘要和关键字

\begin{abstract}
  在不稳定神经网络中,梯度爆炸和梯度消失问题限制了反向传播算法的有效性。随着网络层数和复杂度增加,梯度可能会呈指数级增长或衰亡,导致训练过程中数值不稳定,模型性能下降。本文回顾了不稳定神经网络的理论基础,包括李雅普诺夫谱和李雅普诺夫向量的概念,用于描述系统的动态特性和稳定性。伴随李雅普诺夫谱和对偶性的概念对于解决梯度爆炸问题很重要。

  传统反向传播算法中,梯度爆炸问题的解决方法包括梯度裁剪和正则化技术,但在不稳定神经网络中效果有限。为了克服这个挑战,本文提出了一种基于伴随阴影的新反向传播方法,利用伴随李雅普诺夫谱的信息来调整梯度的传播路径和强度,有效地缓解梯度爆炸问题。同时,介绍了核微分方法,通过引入核函数平滑梯度计算,提高了计算的稳定性和准确性。

  本文在理论层面分析了传统反向传播算法在不稳定神经网络中的表现和局限性,强调了梯度爆炸问题对参数更新和模型训练的影响。基于伴随阴影的反向传播方法重新定义了梯度更新规则,并通过实验验证了其在不同类型不稳定神经网络中的有效性。实验结果表明,该方法显著减小梯度爆炸的影响,提升了模型的收敛速度和性能稳定性。

  为了验证方法的广泛适用性,本文将核微分方法与伴随阴影技术相结合,构建了一种混合优化算法。实验结果显示,与传统方法相比,新的混合优化算法在训练速度、收敛性和最终模型性能方面有显著提升。这表明核微分方法在处理梯度爆炸问题时提供了额外的平滑效果,使得梯度更新过程更加稳定。

  综上所述,本文通过理论分析和实验验证,提出了一种创新的解决不稳定神经网络中梯度爆炸问题的方法。基于伴随阴影的反向传播方法和核微分方法的结合为未来研究和应用提供了新的方向和思路。这些研究结果不仅加深了对不稳定神经网络动态特性的理解,也为改进反向传播算法提供了新的工具和方法。


  % 关键词用“英文逗号”分隔,输出时会自动处理为正确的分隔符
  \thusetup{
    keywords = {神经网络, 反向传播, 李雅普诺夫谱},
  }
\end{abstract}

\begin{abstract*}
  In unstable neural networks, the gradient explosion and gradient vanishing problems limit the effectiveness of backpropagation algorithms. As the number of network layers and complexity increase, the gradient may grow exponentially or decay, resulting in numerical instability during training and degraded model performance. This paper reviews the theoretical foundations of unstable neural networks, including the concepts of Lyapunov spectrum and Lyapunov vector, which are used to describe the dynamic characteristics and stability of the system. The concepts of adjoint Lyapunov spectrum and duality are important for solving the gradient explosion problem.

  In traditional backpropagation algorithms, solutions to the gradient explosion problem include gradient clipping and regularization techniques, but they have limited effects in unstable neural networks. To overcome this challenge, this paper proposes a new backpropagation method based on adjoint shadowing, which uses the information of the adjoint Lyapunov spectrum to adjust the propagation path and intensity of the gradient, effectively alleviating the gradient explosion problem. At the same time, the kernel differentiation method is introduced, and the stability and accuracy of the calculation are improved by introducing kernel functions to smooth the gradient calculation.

  This paper analyzes the performance and limitations of traditional backpropagation algorithms in unstable neural networks at the theoretical level, and emphasizes the impact of the gradient explosion problem on parameter update and model training. The back propagation method based on adjoint shadow redefines the gradient update rule and verifies its effectiveness in different types of unstable neural networks through experiments. Experimental results show that this method significantly reduces the impact of gradient explosion and improves the convergence speed and performance stability of the model.

  In order to verify the wide applicability of the method, this paper combines the kernel differential method with the adjoint shadow technology to construct a hybrid optimization algorithm. Experimental results show that compared with the traditional method, the new hybrid optimization algorithm has significant improvements in training speed, convergence and final model performance. This shows that the kernel differential method provides an additional smoothing effect when dealing with the gradient explosion problem, making the gradient update process more stable.

  In summary, this paper proposes an innovative method to solve the gradient explosion problem in unstable neural networks through theoretical analysis and experimental verification. The combination of the back propagation method based on adjoint shadow and the kernel differential method provides new directions and ideas for future research and application. These research results not only deepen the understanding of the dynamic characteristics of unstable neural networks, but also provide new tools and methods for improving the back propagation algorithm.

  % Use comma as separator when inputting
  \thusetup{
    keywords* = {Neural Network, Backpropagation, Lyapunov Spectrum},
  }
\end{abstract*}
