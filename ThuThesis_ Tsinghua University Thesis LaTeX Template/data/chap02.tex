% !TeX root = ../thuthesis-example.tex

\chapter{李雅普诺夫谱和李雅普诺夫向量}

在这一章中,我们将深入探讨不稳定神经网络的动态特性,重点研究李雅普诺夫谱、李雅普诺夫向量以及伴随李雅普诺夫谱和对偶性. 这些概念和方法在分析神经网络的稳定性和动态行为方面具有重要意义. 

\section{李雅普诺夫谱}

李雅普诺夫谱是描述一个动力系统中轨道对初始条件敏感性的量度. 它通过计算系统中不同方向上的指数增长率,揭示系统的混沌程度和稳定性. 在神经网络中,李雅普诺夫谱可以帮助我们了解网络在训练过程中的动态变化. 

设连续时间动力系统的状态由向量 \(\mathbf{x}(t)\) 描述,其演化方程为:

\[ \frac{d\mathbf{x}}{dt} = \mathbf{f}(\mathbf{x}, t) \quad t\geq 0 \]

李雅普诺夫指数 \(\lambda_i\) 可以通过对系统状态的微小扰动进行分析得到. 首先,我们考虑一个微小扰动 \(\delta \mathbf{x}(t)\),其演化由下式描述:

\[ \frac{d (\delta \mathbf{x})}{dt} = \mathbf{J}(\mathbf{x}, t) \delta \mathbf{x} \]

其中,\(\mathbf{J}(\mathbf{x}, t)\) 是系统的雅可比矩阵,定义为:

\[ \mathbf{J}(\mathbf{x}, t) = \frac{\partial \mathbf{f}}{\partial \mathbf{x}} \]

李雅普诺夫指数通过分析扰动向量 \(\delta \mathbf{x}(t)\) 的指数增长率定义为:

\[ \lambda_i = \lim_{t \to \infty} \frac{1}{t} \ln \frac{||\delta \mathbf{x}_i(t)||}{||\delta \mathbf{x}_i(0)||} \]

对于离散时间系统,李雅普诺夫指数的定义类似,只是将微分方程替换为差分方程. 通过计算系统中不同方向上的李雅普诺夫指数,我们可以得到李雅普诺夫谱,进而分析系统的稳定性和混沌特性. 

神经网络中的李雅普诺夫谱属于离散时间系统,其计算方法如下:

1. 雅可比矩阵计算:在每一层的前向传播和反向传播过程中,计算出相应的雅可比矩阵. 这些矩阵描述了网络参数对输入数据的敏感性. 

\[ \mathbf{J}_l = \frac{\partial \mathbf{a}_l}{\partial \mathbf{a}_{l-1}} \]

其中,\(\mathbf{a}_l\) 是第 \(l\) 层的激活值. 

2. QR 分解:对每一步计算得到的雅可比矩阵进行QR分解,提取出李雅普诺夫指数. QR分解是一种数值稳定的方法,可以有效地处理高维矩阵. 

\[ \mathbf{J}_l = \mathbf{Q}_l \mathbf{R}_l \]

3. 指数累积:在每一次分解之后,累积李雅普诺夫指数的变化,并对这些指数进行归一化处理,以防止数值溢出. 

\[ \lambda_i = \lim_{N \to \infty} \frac{1}{N} \sum_{l=1}^N \ln |r_{ii}(l)| \]

其中,\(r_{ii}(l)\) 是第 \(l\) 步QR分解中矩阵 \(\mathbf{R}_l\) 的对角线元素. 

通过以上步骤,我们可以得到神经网络的李雅普诺夫谱,并据此分析网络的稳定性和动态行为. 

\section{李雅普诺夫向量}

李雅普诺夫向量是与李雅普诺夫指数对应的特征向量,它们描述了系统在各个方向上的扩展或收缩速率. 具体而言,正的李雅普诺夫指数对应的向量表示系统在该方向上具有指数增长的性质,而负的李雅普诺夫指数对应的向量表示系统在该方向上具有指数衰减的性质. 

李雅普诺夫向量在求解李雅普诺夫谱的同时得到,所有李雅普诺夫向量放在一起构成一个正交基,实际上就是 QR 分解的结果 Q 矩阵.

在神经网络的训练过程中,李雅普诺夫向量可以帮助我们识别网络中对输入变化最敏感的方向,从而指导网络参数的调整和优化. 例如,在梯度下降过程中,我们可以利用李雅普诺夫向量来调整学习率,使得网络在每一步更新中更加稳定. 

计算李雅普诺夫向量的步骤如下:

\begin{enumerate}

\item 初始向量设定:选择一个初始向量集合,通常为标准正交基. 
   
\item QR分解迭代:在每一步迭代中,对雅可比矩阵进行QR分解,并更新向量集合. 
   
\item 向量正交化:在每一步迭代后,对向量集合进行正交化处理,以确保向量的独立性和数值稳定性. 

\end{enumerate}

设初始向量为 \(\mathbf{v}_i(0)\),在第 \(l\) 层的QR分解过程中更新为:

\[ \mathbf{v}_i(l) = \mathbf{Q}_l \mathbf{v}_i(l-1) \]

通过以上步骤,我们可以得到与每一个李雅普诺夫指数对应的特征向量集合,从而深入理解神经网络的动态特性. 