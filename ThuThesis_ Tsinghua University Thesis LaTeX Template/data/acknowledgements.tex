% !TeX root = ../thuthesis-example.tex

\begin{acknowledgements}
总觉得来日方长,却不知岁月清浅,时节如流。当我提笔写下致谢时才发现,四年的大学生活即将结束,终于到了该说再见的时候了。四年的旅程,所有的相遇,所有的经历于我而言都是最好的礼物。愿走出校园的我们都会成为会更好的自己。

桃李不言,下自成蹊。在这次综合论文训练中,我最想要感谢的人是我的指导老师,倪昂修老师。相遇就是缘分,是良师亦是朋友,我想不到用什么华丽的语言来形容他,但是说起在做毕设和写论文过程中对我帮助最大的人,我第一时间想到的就是倪老师,从选题到中期,再到最终成文,他一直在很认真的指导我完成毕设和论文,并给出自己的建议,对于提出的的问题能够及时回复,除此之外,他还会关心我们的生活和工作,并给予一定的帮助和引导,是一位非常尽职尽责的老师涓涓师恩,铭记于心,感谢他帮助我完成了毕设和论文。我亦对于参与答辩工作的老师十分感激,感谢你们拨冗予以指导意见,让答辩对我显得尤为珍贵。

其次,我想感谢的是我的家人。我的家庭并非大富大贵之家,父母都是兢兢业业的教师,二十年来,对我的教育一直是包容胜过苛责,理解多于否定,在我心中,他们就是这个世界上最伟大的人,他们给了我生命,教会我成长,尊重我的选择,给予我无限的包容和关怀,是我最坚强的后盾。春晖寸草,难以回报,希望父母平安喜乐。

也感谢我的朋友,感谢你们在我写论文和毕设时给予的帮助。是你们陪伴我走过这四年的大学生涯,让平淡的生活增加了很多趣味,在我需要帮助时总是第-时间出现在我身边,让我在这四年感受到了很多的温暖和快乐,尤其感谢夏斐然同学,在大学四年里给我的生活带去了无穷乐趣。山河不足重,重在遇知己,祝大家前程似锦,在各自的领域闪闪发光。

最后我想感谢自己。我想对过去平凡且努力的自己说一声谢谢,这一路走来谈不上筚路蓝缕,但是也绝非易事,最让我引以为傲的事情就是一直在做自己,我们都应该活成自己喜欢的样子,做自己喜欢的事情,和喜欢的人交往,接受平凡的自己,也接受不完美的自己。

在走入社会后,希望自己永保初心,自由独立自信勇敢、不必羡慕谁,也不依附谁,做一个心中有光的人。宇宙山河烂漫,人间点滴温暖都值得我们继续前进。

行文至此,落笔为终。可以回头看,但不能走回头路,追风赶月莫停留,平芜尽处是春山,彼方尚有荣光在,愿我们前路漫漫亦灿灿。

\end{acknowledgements}
